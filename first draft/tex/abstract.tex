% !TEX root = ../thesis.tex

\begin{abstract}
  随着信息过载时代的到来,用户想要从海量信息中找到自己感兴趣的信息变得十分困难,而推荐系统可以通过分析用户历史行为,挖掘用户行为模式,找出用户行为特征并对用户偏好做出预测。真实世界的用户行为数据通常存在很多的问题,诸如数据的稀疏性,隐性反馈行为特征,冷启动问题等等,而解决这些问题也显得尤为重要。本文将基于真实世界的一个用户行为数据集,构建一个为用户推荐动漫的推荐系统,通过多种基本的推荐召回算法实现不同的推荐目的,再利用机器学习技术把多种算法的结果组合起来进行排序以获得更好的效果。同时,采用KNN填充和基于邻域的评分预测有效地解决了稀疏性问题,并通过结合隐式反馈特征和显式反馈特征的方法,选取更合适的样本构造规则。

  \keywords{推荐系统; 协同过滤; 机器学习; 缺失值; 隐反馈特征}
\end{abstract}

\begin{enabstract}
  With the arrival of the era of information overload, it becomes quite difficult for users to find information they are interested in from the huge amount of information. However, the recommender system can find out the user behavior characteristics through analyzing the users' historical behavior and mining user behavior patterns, and then make predictions on users' preferences. There are many problems in real-world user behavior data, such as data sparsity, implicit feedback behavior features, cold start problems, etc, and it is also important to solve these problems. In this thesis, we will build a recommender system based on a real-world user behavior data set, to recommend animation for users. We will choose a variety of basic recommendation recall algorithms to achieve different recommendation goals, and then make use of machine learning technology to sort the combination result of these algorithms for a better behavior. As for the existing problems mentioned above, KNN filtering and neighborhood-based scoring prediciton are used to solve the problem of score data sparsity effectively, and a more appropriate sample construction rule is selected by taking both explicit and implict feedback feature into consideration.

  \enkeywords{Recommender Systems; Collaborative Filtering; Machine Learning; Missing Value; Implicit Feedback Features}
\end{enabstract}
