% !TEX root = ../thesis.tex

\chapter{数据准备与实验设计}
  \section{数据集的简介与清理}
  本文选用Kaggle上公开的一个真实世界数据集,是作者从网站\url{MyAnimeList.net}中爬取到的用户将动漫添加到自己的list中、观看动漫、给动漫评分等一系列行为的数据。下载下来的数据集分成三个部分:用户信息、动漫信息、及用户给动漫打分的信息,作者已经对数据做了一定清洗处理,清洗后的数据集约包含10.87万用户对6600部动漫的3000万条打分记录。用户信息和动漫信息表主要是关于用户和动漫这两个维度的一些固有属性;对于打分行为表,一些比较特殊重要的特征如下:
  \begin{itemize}
    \item my\_watched\_episodes:已观看该动漫集数;
    \item my\_score:打分,取值为0到10之间的整数;
    \item my\_status:观看状态,主要取值有:$\{1:\text{watching},\;2:\text{completed},\;3:\text{on hold},\;4:\text{dropped},\;6:\text{plan to watch}\}$;
    \item my\_last\_updated:最后一次更新状态的日期。
  \end{itemize}

  在作者清洗的基础上,额外进行的清洗工作如下:
  \begin{enumerate}
    \item 删除last\_update\_date字段取值为"1970-01-01"的异常样本;
    \item 删除my\_status字段取值不属于$\{1,2,3,4,6\}$的样本(其他取值具体含义不明,且样本量很少);
    \item 删除last\_update\_date字段取值早于该动漫的最早上映日期的样本;
    \item 结合用户观看集数与该动漫总集数,修正my\_status取值。具体地,若观看集数等于总集数且非零,则令my\_status = 2;若观看集数不等于零,且原本的my\_status = 6(即准备观看),则修改my\_status = 1(观看中);
    \item 最后,由于原始数据集的样本量太大(3000万行,11万用户),单机情况下程序执行有很大困难,因此决定对原始数据集进行抽样。抽样的方法是对11万用户随机抽样5\%,约5500名用户,抽样后的打分行为数据集大小约为151万行。抽样的合理性在后续章节会进行实验分析。
  \end{enumerate}

  \section{训练集和测试集划分}
  本实验目的是利用这些用户行为数据,构建一个为用户推荐他可能感兴趣动漫的推荐系统,这种推荐属于TopN推荐而非评分推荐。具体地,这种推荐的目的是预测用户是否会看某部动漫,而不是预测用户对该动漫打多少分。

  打分行为数据集中的字段last\_update\_date,其含义是用户更新状态的最后一次时间。表~\ref{tab:date_distribution}是对原始数据集和抽样数据集中该字段的分位数分布统计:
  % Table generated by Excel2LaTeX from sheet 'Sheet1'
  \begin{table}[htbp]
    \centering
    \caption{原始和抽样数据集中的日期分布}
    %\textbf{表3.1}~~原始和抽样数据集中的日期分布
      \begin{tabular}{lrr}
      \toprule
      quantile & raw dataset & sampling dataset \\
      \midrule
      10\% & 2009-08-02 & 2009-08-12 \\
      20\% & 2010-12-18 & 2010-12-26 \\
      30\% & 2012-04-17 & 2012-04-16 \\
      40\% & 2013-05-11 & 2013-04-23 \\
      50\% & 2014-04-10 & 2014-03-31 \\
      60\% & 2015-03-08 & 2015-02-26 \\
      70\% & 2015-12-28 & 2015-12-20 \\
      80\% & 2016-09-28 & 2016-09-23 \\
      90\% & 2017-07-24 & 2017-07-16 \\
      \bottomrule
      \end{tabular}%
    \label{tab:date_distribution}%
  \end{table}%

  根据我们要推荐的“物品”,即动漫其本身的季节性特点,通常分为一年四个季度上映,上映日期分别在1月,4月,7月和10月。因此考虑取时间划分节点为"2017-06-30",即last\_update\_date取值小于"2017-06-30"的样本为训练集,剩下的样本为测试集。这样划分下来的训练集/测试集比例大约为9:1。

  \section{正负样本构造}
  划分完训练测试集后,还需要设置样本标签。一方面,各召回算法所利用的“用户历史上喜欢过的动漫”数据,其中的“喜欢”和“不喜欢”即是由样本标签表示;另一方面,应用机器学习技术对召回结果进行重排序是一个有监督模型,因此也需要样本标签。对于TopN推荐,比较合适的重排序是选择一个二分类模型,最后根据模型输出的概率值排序。

  测试集标签设置比较容易。TopN推荐关心的是用户是否观看了系统所推荐的结果,所以对于任意测试样本,只要my\_watched\_episodes不等于零(表示用户看过了这一部动漫),则令样本标签$y=1$,否则$y=0$。

  训练集标签设置相对复杂。从模型训练角度看,正样本应表明用户喜欢看这部动漫,而不仅是“看过”,因此标签的设置依赖于用户对动漫的打分my\_score,打分越高,用户的喜欢程度越大。因为是二分类问题,样本标签$y\in\{0,1\}$。具体的方法是对分数取一个阈值,令打分高于阈值的样本$y=1$,否则$y=0$。表~\ref{tab:status_score_distribution}是数据集中各个status对应的打分分布情况,以及各个分值在网站中的含义:

  \begin{table}[htbp]
    \centering
    \caption{各status对应打分分布情况}
    %\textbf{表3.2}~~各status对应打分分布情况
    \resizebox{\textwidth}{!}{
      \begin{tabular}{lrrrrrr}
      \toprule
      my\_score & \multicolumn{1}{l}{meaning} & 1    & 2    & 3    & 4    & 6 \\
      \midrule
      \multicolumn{1}{r}{1} & \multicolumn{1}{l}{Appaling} & 0.30\% & 0.40\% & 0.30\% & 4.50\% & 8.80\% \\
      \multicolumn{1}{r}{2} & \multicolumn{1}{l}{Horrible} & 0.20\% & 0.50\% & 0.20\% & 4.90\% & 0.10\% \\
      \multicolumn{1}{r}{3} & \multicolumn{1}{l}{Vey bad} & 0.30\% & 0.80\% & 0.50\% & 7.50\% & 0.30\% \\
      \multicolumn{1}{r}{4} & \multicolumn{1}{l}{Bad} & 0.60\% & 1.90\% & 1.40\% & 15.70\% & 0.60\% \\
      \multicolumn{1}{r}{5} & \multicolumn{1}{l}{Average} & 2.90\% & 4.70\% & 6.10\% & 24.20\% & 4.90\% \\
      \multicolumn{1}{r}{6} & \multicolumn{1}{l}{Fine} & 8.30\% & 10.60\% & 16.40\% & 20.00\% & 5.70\% \\
      \multicolumn{1}{r}{7} & \multicolumn{1}{l}{Good} & 21.80\% & 22.30\% & 30.60\% & 14.20\% & 11.80\% \\
      \multicolumn{1}{r}{8} & \multicolumn{1}{l}{Very good} & 27.20\% & 26.50\% & 25.20\% & 5.80\% & 19.20\% \\
      \multicolumn{1}{r}{9} & \multicolumn{1}{l}{Great} & 21.00\% & 19.00\% & 12.10\% & 1.90\% & 13.90\% \\
      \multicolumn{1}{r}{10} & \multicolumn{1}{l}{Masterpiece} & 17.60\% & 13.40\% & 7.20\% & 1.20\% & 34.80\% \\
      proportion &      & 2.37\% & 89.07\% & 2.24\% & 5.92\% & 0.40\% \\
      zero propotion &      & 64.80\% & 12.40\% & 64.60\% & 48.70\% & 98.60\% \\
      \bottomrule
      \end{tabular}}%
    \label{tab:status_score_distribution}%
  \end{table}%

  从表~\ref{tab:status_score_distribution}中可以得出:
  \begin{itemize}
    \item 数据集打分缺失比例较高,特别是my\_status=6的样本几乎全是0分;
    \item 各status平均打分大约在7-8分左右,除去my\_status=4比较低。
  \end{itemize}

  结合上面的结论,训练样本标签设置方法如下:
  \begin{enumerate}
    \item 阈值初设为8分,高于8分为正样本,低于8分且打分非零的为负样本(8分的理由是略高于各状态平均打分);
    \item my\_status=4的零分样本也是负样本(状态4对应放弃观看);
    \item my\_status=1,2,3的零分样本剔除出训练集(后续填充缺失时使用);
    \item my\_status=6的零分样本剔除出训练集(状态6几乎都是0分)。
  \end{enumerate}

  处理之后训练集最终大小约为90万行,正负样本比约为$1.25:1$。

  \section{用户与物品画像表设计}
  原始的用户信息和动漫信息数据集含有许多的类别特征、文本特征。为了能够在后面的机器学习模型中应用这些数据,需要做一定的预处理。对于用户画像特征,主要做的处理有:
  \begin{itemize}
    \item 根据birth\_date,计算用户的年龄age(用"2017-06-30"减去birth\_date,结果向下取整);
    \item 统计每个用户在训练集中各status对应的观测数;
    \item 统计每个用户在训练集中“喜欢”的动漫数(即$y=1$的记录数);
    \item 统计每个用户喜欢的动漫中,各source(原作类型)、各rating(适宜人群)及各genre(流派风格)的占比;
    \item 对之前剔除出的my\_status=6的样本,同样统计每个用户对应上述各特征占比。
  \end{itemize}

  对于物品画像特征,主要做的处理有:
  \begin{itemize}
    \item 将打分人数scored\_by和成员数members合并成一个新字段scored\_ratio;
    \item members的数值区间太大,将其分箱成6个区间,对应不同的流行度;
    \item 适当扩充type字段,把type=TV扩充成长篇、半年番、季番、短篇以及泡面番;
    \item source字段适当精简合并成9类;
    \item genre字段挑选出有代表性的20个流派风格。由于同一部动漫可以同时包含多个genre,还需要对genre做One Hot处理。
  \end{itemize}

  \section{实验设计}
  本文后续实验分为下面几个步骤:
  \begin{enumerate}
    \item 基于随机抽样后的打分行为数据集,采用四种不同召回算法,并比较分析这四种算法的效果差异。然后利用Xgboost
    模型对召回结果进行打分重排,输出最终结果。比较排序后的模型效果和排序前的模型效果差异。
    \item 对随机抽样的可行性分析,设计实验比较抽样、全量与增量训练彼此之间的效果差异。
    \item 对数据稀疏问题,采用KNN填补和基于邻域的评分预测方法,比较填补缺失后的召回算法效果与填补前的差异。
    \item 对隐反馈特征的利用,包括my\_watched\_episodes, my\_status等,将其和my\_score综合考虑来确定样本的标签,并比较模型的效果差异。
  \end{enumerate}