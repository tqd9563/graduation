% !TEX root = ../thesis.tex

\chapter{绪论}
  \section{研究背景及意义}
  随着互联网的快速普及和信息技术的高速发展,信息资源正呈现着爆炸式的增长,不论何时何地,人们都可以非常方便地通过互联网来获取信息。2018年天猫“双11”活动,每秒创建订单峰值高达惊人的49.1万笔;今年8月,网易的二季度财报披露网易云音乐的总用户数已经超过了8亿,而这一数字在两年前是4亿,三年前仅是2亿。明显地,人们正在从原先的信息匮乏时代,逐渐地步入信息过载时代。而这个信息过载的时代,不论是对于信息的生产方,还是对于我们这样普通的信息消费方,都带来了巨大的挑战。一方面,对于信息的消费方普通用户而言,面对这海量的数以万计甚至百万计的物品,倘若没有一个明确的需求目标,想要找到对我们有用的,或是我们感兴趣的物品会变得十分困难。而另一方面,对于信息的生产提供方商家而言,如何从这海量的商品库存中,为用户准确地推荐可能符合他们需求的物品,从而自己能够获得收益,这也是十分困难的。在这样的大环境下,推荐系统(Recommender Systems, RS)便应运而生,它主要解决的就是上面所提到的用户-商家双向问题。以电子商务领域为例,推荐系统一方面可以为用户产生个性化的推荐,使得消费者发现对自己有价值的或是感兴趣的物品,进而促进消费者购买欲望,提高其对商家的粘性;另一方面,也有助于商家对用户进行需求分析,从而准确地把握市场需求,带来盈利\cite{卢棪2016协同过滤推荐系统研究及其应用},从而实现用户和商家的互利共赢。

  个性化推荐系统,与我们熟知的搜索引擎类似,都是帮助用户发现信息的一种工具。两者最大的区别在于是否需要用户进行明确输入\cite{2012推荐系统实践}:搜索引擎需要用户输入关键词,然后才能进行搜索,如果想要的东西不能很明确的通过一个或多个关键词进行描述的话,搜索引擎便无法很好工作;而推荐系统不需要用户提供显式的输入,它的核心原理是挖掘用户的行为模式,通过对海量的用户历史行为数据进行分析,然后建立模型,通过模型来预测每个用户的兴趣爱好,最终为不同的用户推荐不同的“个性化”物品。也正是因为这种工作原理,推荐系统通常不会像搜索引擎一样独立的成为一个网站,如百度、Google等等,而是会存在于每个网站的不同应用或是模块的背后,为这些应用和模块的正常工作而服务。

  目前,推荐系统已经在国内外许许多多的领域内有着非常广泛而成功的应用:电子商务网站,如淘宝、京东的商品个性化推荐页面;视频网站,如Youtube首页的推荐视频;社交网络,如新浪微博的热门推荐;音乐电台,如网易云音乐的每日歌曲推荐/私人FM等等。在这些领域内,推荐系统的实际应用也为公司带来了庞大的经济收益。著名的北美在线视频服务提供商Netflix,其自己估计每年通过个性化推荐系统为自身业务节省了约10亿美金\cite{gomez2016netflix};著名电子商务网站Amazon也曾披露,其销售额有20\%-30\%是来自推荐系统。正因如此,对推荐系统的研究在学术界也是越来越受到学者们的关注,并成为了一个比较热门的方向。同时由于其和不同的学科研究领域,比如数据挖掘、机器学习、人工智能等都有着一定的联系,这种交叉学科的性质使得推荐系统正在飞速的向前发展。在当下这个信息爆炸的时代,研究推荐系统是很有意义的,对互联网生态的发展也有着一定的意义。

  \section{国内外研究现状}
  1994年,明尼苏达大学的GroupLens研究组设计了第一个自动化的新闻推荐系统GroupLens\cite{resnick1994grouplens}。这个系统首次提出了协同过滤的思想,并且为后世的推荐问题建立了一个形式化的范式,可以算是最早的推荐系统。1997年,Resnick等人\cite{resnick1997recommender}首次提出推荐系统(Recommender System,RS)一词,自此,推荐系统一词被广泛引用,并且推荐系统开始成为一个重要的研究领域。1998年,著名的个性化商城亚马逊(Amazon)提出了基于物品的协同过滤算法(Item-based Collaborative Filtering),在此之前所普遍使用的协同过滤算法都是基于用户的(User-based CF),而Amazon提出的这个新算法的实际效果非常好。2003年,Amazon在IEEE Internet Computing上公开了这个item-CF算法\cite{linden2003amazon},带来了广泛的关注和使用,包括YouTube、Netflix等著名海外公司。2005年Adomavic{}ius等人的综述论文 将推荐系统分为3个主要类别,即基于内容的推荐、基于协同过滤的推荐和混合推荐的方法,并提出了未来可能的主要研究方向\cite{adomavicius2005toward}。

  到了2006年,一个大事件将推荐系统的研究推向了快速发展的高潮阶段:Netflix宣布了一项竞赛,第一个能将现有推荐算法的准确度提升10\%以上的参赛者将获得100万美元的奖金。这个比赛在学术界和工业界引起了很大的关注,吸引了来自186个国家和地区的超过4万支队伍参赛。而在之后的那几年,许多经典的推荐算法被提出,比如:Koren等人对利用矩阵分解(Matrix Factorization, MF)实现协同过滤的现有技术进行了综述\cite{koren2009matrix},包括基本MF原理,以及包含隐反馈、时序动态等特殊元素的MF等;Steffen Rendle等人在2009年提出了一种“基于贝叶斯后验优化”的个性化排序算法BPR\cite{rendle2009bpr},它是采用pairwise训练的一种纯粹的排序算法。

  近年来随着人工智能的火爆,机器学习和深度学习技术也被广泛运用到了推荐系统的排序场景中。Google在2016年最新发布的模型Wide\&Deep\cite{cheng2016wide}就是综合应用了机器学习和深度学习的产物。它将逻辑回归(LR)与深度神经网络(DNN)结合在一起,既发挥了LR强解释性、高效易规模化的优势,又补充了DNN的强泛化与自动特征组合能力。这个方法应用在Google play的推荐场景中并获得了良好的效果;另一个DeepFM模型\cite{guo2017deepfm}则是将传统的因子分解机FM和DNN结合在一起,用来在CTR预估中挖掘构造有用的高级交叉特征。

  除了基本的推荐算法外,关于推荐系统的一些其他方面,比如冷启动问题、缺失数据的填充、隐反馈信息的利用等等,也涌现出不少优秀的研究成果。比如用隐语义模型(LFM)来解决隐性反馈数据的协同过滤问题\cite{hu2008collaborative};用完整数据的协同过滤来预测缺失的得分信息\cite{ren2012efficient};通过最大化TOPK曲线下面积,将缺失得分统一填充为一个较低的数值\cite{steck2010training};利用一种简单的挖掘关联属性的方法来处理Item-based CF中的完全物品冷启动问题\cite{zhang2019addressing}等等。

  \section{研究问题与内容}
  真实世界的用户行为数据集通常存在着以下几个问题:
  \begin{enumerate}
    \item 数据的稀疏性。由于用户和物品的基数庞大,每个用户只会对一小部分的物品有过行为,如给电影打分,购买物品等等,而对于剩下的大量的其他物品,该用户都没有过行为。这样导致我们生成的用户-物品UI矩阵就是一个稀疏矩阵,这会给某些推荐算法预测用户喜好造成困难,导致推荐效果不理想。

    \item 用户的显性与隐性行为。显性行为指的是可以明确反映用户对物品的喜好程度的行为,最有代表性的的显性行为就是打分,打分高说明用户喜欢这个物品,打分低则说明用户不喜欢这个物品。与之对应的,隐性行为就是那些不能明确反映用户喜好程度的行为。以电子商务网站为例,用户给商品打分属于显性行为,而诸如用户对商品的浏览、点击等行为就属于隐性行为。在真实数据集中,隐性行为要远多于显性行为,有效的利用隐性行为特征可以提高推荐的效果。

    \item 冷启动问题,具体地又分为用户的冷启动,以及物品的冷启动。其中,用户的冷启动指的是如何对新用户进行推荐,因为新用户是没有历史行为数据可供我们的推荐系统进行挖掘分析的;而物品的冷启动主要解决的是如何把一个新的物品推荐给用户,因为这个新的物品在所有的历史行为数据中肯定是没有出现过的。
  \end{enumerate}

  本文选取了网络上公开的一个真实世界中的数据集,这个数据集也有着上面提到的这些问题。基于该数据集的基础上,首先探讨比较了多种召回算法的效果与优缺点;然后将机器学习技术应用于推荐模型的排序阶段,对召回结果进行精排,比较精排后与精排前的效果;然后对于稀疏打分问题,采用KNN填充和基于邻域的评分预测的方法来填补缺失得分;对于隐性反馈行为,采用隐反馈特征和显反馈特征相结合的方法来生成样本标签,基于网格搜索的方法来选择合适的参数组合,并比较处理前后的模型效果。