% !TEX root = ../thesis.tex

\begin{summary}
推荐系统如今已广泛地应用于许许多多的互联网产品中,并有着非常可观的成果。它既能准确的分析用户的兴趣偏好,为用户提供有价值的物品;又能帮助供应商家将物品展现在需要它的用户面前,牟取商业利益。也因此,越来越多的学者和研究者开始研究推荐系统,并提出了各种各样的推荐算法。除了研究推荐算法以外,推荐系统还有许多其他的问题值得研究讨论,比如冷启动问题、可解释性与多样性、数据的稀疏性问题等等。

本文基于真实世界的一个数据集,着手构建一个为用户推荐动漫的推荐系统。根据不同的推荐目的,采用了四种基本的推荐算法用于推荐召回阶段,它们有着各自的优缺点。出于模型融合的思想,利用机器学习模型对这些召回结果整合后进行重排序,并显著提高了推荐性能。由于数据集的容量非常大,因此模型是用的采样数据集训练,后面通过实验验证了这种采样方法的可靠性。对于真实世界数据集存在的诸多问题,尝试了基于KNN的填充和基于邻域的评分预测填充两种方法来解决数据稀疏问题,并将显式反馈特征与隐式反馈特征结合在一起作为样本标签构造的规则,通过grid-search来寻找更优的特征阈值参数组合。结果发现这些方法都可以一定程度的提高模型最终的推荐效果。

除了上述所做工作以外,还可以有以下一些优化的方向:
\begin{enumerate}
	\item 数据稀疏性问题,文中用的KNN填充和基于物品邻域的评分预测填充方法虽然都能改善模型效果,不过其各自对召回算法的效果影响不尽相同,因此还可以尝试对些方法加权集成。或是寻找其他更好的填充缺失的算法;
	\item 隐反馈特征问题,除了可以把显式特征和隐式特征结合在一起来构造样本标签外,还可以尝试根据隐反馈特征对样本进行加权,或者是其他的一些构造样本的方法;
\end{enumerate}
\end{summary}
